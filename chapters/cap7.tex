\chapter{Listas duplamente encadeadas}

No capítulo anterior, vimos que a lista encadeada simples é uma estrutura recursiva que contém pelo menos dois campos: um valor (val) e um ponteiro para o próximo nó. As listas encadeadas simples permitem a construção de lista com tamanho dinâmico sem acesso randômico. Essa estrutura permite um gerenciamento de memória mais flexível. Além disso, elas permitem que as operações de inserção e deleção possam ser feita em tempo constante. 


\begin{center}
\begin{tabular}{|l|ll|}
\hline
\multicolumn{1}{|c|}{\multirow{2}{*}{Estrutura de Dados}} & \multicolumn{2}{l|}{Inserção e deleção} \\ \cline{2-3} 
\multicolumn{1}{|c|}{}                                    & \multicolumn{1}{l|}{Início}    & Fim    \\ \hline
Lista Sequencial Dinâmica                                 & \multicolumn{1}{l|}{O(n)}      & O(1)   \\ \hline
Lista encadeada simples com nó cabeça                     & \multicolumn{1}{l|}{O(1)}      & O(n)   \\ \hline
Lista encadeada simples com nó cabeça e nó rabo           & \multicolumn{1}{l|}{O(1)}      & O(1)   \\ \hline
\end{tabular}
\end{center}

A principal desvantagem das listas encadeadas é permitir apenas a travessia em apenas uma direção. Além disso, só podemos inserir um novo elemento após um dado nó. Essa restrição pode tornar alguns problemas mais complicados.



Considere a seguinte definição do nó de uma lista encadeada simples:
\begin{listing}[!ht]
\caption{Definição do nó de uma lista encadeada simples}

\begin{minted}{C++}
template <typename T>
class Node {
    public:
    T val;
    Node *next;
    Node() : val( T() ), next(nullptr) {}
    Node(T x) : val(x), next(nullptr) {}
    Node(T x, Node *next) : val(x), next(next) {} 
};

\end{minted}
\end{listing}


Considere que você recebe o ponteiro para nó cabeça de uma lista encadeada simples e um nó da lista a ser removido. Para remover esse nó da lista precisamos encontrar o nó anterior a ele para que esse nó aponte para o próximo do nó removido.

\newpage 
\begin{listing}[!ht]
\caption{Deleção de um nó em uma lista encadeada simples}

\begin{minted}{C++}

template <typename T>
Node<T> * deleteNode(Node<T> * head, Node<T> * node){
    if(head == nullptr )
        return head;
    else if(head == node){
        auto ptr = head->next;
        delete head;
        return ptr;
    }else{
        auto prev = head;
        while( prev->next != node){
            prev = prev->next;
        }
        prev->next = node->next;
        delete node;
        return head;
    }
}
\end{minted}
\end{listing}

Considere agora a  seguinte definição de uma lista encadeada duplamente encadeada:


\begin{listing}[!ht]
\caption{Definição do nó de uma lista duplamente encadeada}
\begin{minted}{C++}
template <typename T>
class Node {
    public:
    T val;
    Node *next;
    Node *prev;
    Node() : val( T()  ), next(nullptr), prev(nullptr) {}
    Node(T x) : val(x), next(nullptr), prev(nullptr) {}
    Node(T x, Node *prev, Node *next) : val(x), prev(prev), next(next) {} 
};

\end{minted}
\end{listing}

O mesmo problema pode ser resolvido facilmente na lista duplamente encadeada. Como pode ser visto na Listagem \ref{listing::DeleteNodeDouble}.

\begin{listing}[!ht]
\caption{Deleção de um nó uma lista duplamente encadeada}
\label{listing::DeleteNodeDouble}
\begin{minted}{C++}
template <typename T>
Node<T> * deleteNode(Node<T> * head, Node<T> * node)
{
    node->prev->next = node->next;
    node->next->prev = node->prev;
    delete node;

    return head;
}
\end{minted}
\end{listing}



\section{Recursão em listas duplamente encadeada}

\subsection{Busca de um elemento}
Dado o nó cabeça de uma lista duplamente encadeada. Sua tarefa é
verificar se um elemento x está presente na lista duplamente encadeada.

Matematicamente, podemos expressar esse problema recursivamente da seguinte maneira:


\begin{equation}
search(head, x) = 
\begin{cases}
false           & , head == nullptr\\
true            & , head->val == x\\
search(head->next, x) & , \text{caso contrário}\\

\end{cases}
\end{equation}

\textbf{Caso Base}

Neste problema, temos dois casos bases:

\begin{enumerate}
    \item Quando a lista duplamente encadeada está vazia (head == nullptr). Neste caso, devolvemos o valor booleano falso.
    \item Quando o valor armazenado na cabeça da lista é igual ao valor x buscado (head->val == x). Neste caso, devolvemos o valor booleano true. 
\end{enumerate}


\textbf{Caso Recursivo}

Neste caso, podemos reduzir o problema para um caso menor, ou seja, buscar o elemento x na lista duplamente encadeada cuja cabeça é head->next.


\begin{listing}[!ht]
\caption{Busca de um elemento em uma lista duplamente encadeada}
\label{listing::DeleteNodeDouble}
\begin{minted}{C++}
template <typename T>
bool * searchNode(Node<T> * head, T x)
{
    if(head == nullptr)
        return false;
    else if(head->val == x)
        return true;
    else   
        return searchNode(head->next, x);
}
\end{minted}
\end{listing}


\subsection{Remover um elemento de uma lista encadeada}

Dado o nó cabeça de uma lista duplamente encadeada. Sua tarefa é
remover o nó com o valor x da lista duplamente encadeada.

Matematicamente, podemos expressar esse problema recursivamente da seguinte maneira:


\begin{equation}
remove\_element(head, x) = 
\begin{cases}
head            & , head == nullptr\\
head->next      & , head->val == x\\
ptr = remove\_element(head->next, x) & , \text{caso contrário}\\
head->next = ptr & , \text{caso contrário}\\
ptr->prev = head & , \text{caso contrário}\\
\end{cases}
\end{equation}

\textbf{Caso Base}

Neste problema, temos dois casos bases:

\begin{enumerate}
    \item Quando a lista duplamente encadeada está vazia (head == nullptr). Neste caso, devolvemos nullptr.
    \item Quando o valor armazenado na cabeça da lista é igual ao valor a ser removido. Neste caso, devolvemos head->next.  
\end{enumerate}

\textbf{Caso Recursivo}

Quando o valor armazenado na cabeça da lista é diferente ao valor a ser removido, reduziremos o problema a remove o elemento x da lista duplamente encadeada com a cabeça em head->next com resultado ptr. Em seguida, reconstruiremos a lista duplamente encadeada inicial com as seguintes operações:

\begin{enumerate}
    \item head->next = ptr
    \item ptr->prev  = head, se ptr != nullptr
\end{enumerate}

\begin{listing}[!ht]
\caption{Remover um elemento de uma lista duplamente encadeada}
\begin{minted}{C++}
template <typename T>
Node<T> * remove_element(Node<T> * head, T x){
    if(head == nullptr)
        return nullptr;
    else
    {
        if(head->val == x)
        {
           auto ptr  = head->next;
           delete head;
           return ptr;
        }
        else
        {
            head->next = remove_element(head->next, x);
            if( head->next )
                head->next->prev = head;
            return head;
        }
    }
}
\end{minted}
\end{listing}

\section{Lista Duplamente Encadeada com descritores}

Um descritor de uma estrutura de dados é uma informação que ajuda a descrever a nossa estrutura e permite que algumas operações possam ser realizada de maneira mais eficiente.

A implementação da lista duplamente encadeada apresentada aqui possui os seguintes descritores:
\begin{itemize}
    \item Um nó sentinela chamado head que aponta para o primeiro elemento da nossa lista encadeada.
    \item Um nó sentinela chamado past\_last que é o próximo do último elemento da lista encadeada.
    \item a variável inteira count representando o tamanho da lista encadeada.
\end{itemize}

Observe que com esses descritores podemos facilmente adicionar um elemento no ínicio e fim da lista encadeada e saber o tamanho da lista encadeada.

Além disso, temos duas classes aninhadas:

\begin{itemize}
    \item A classe Node que descreve o nó de uma lista encadeada.
    \item A classe Iterator utilizada para navegar pela lista encadeada sem expor sua representação interna.
\end{itemize}



\begin{listing}[!ht]
\caption{Definição da Lista duplamente encadeada com descritores}
\begin{minted}{C++}
template <typename T> 
class List;

template <typename T> 
class List{
    private:
        class Node;
        Node * head;
        Node * past_last;
        int count;
    public:        
        class Iterator;
        List();
        int size();
        void insert_front(T x);
        void insert_back(T x);
        void remove_front();
        void remove_back();
        T & front();
        T & back();
        auto begin();
        auto end();
        void insert_before(Iterator it, T x);
        void erase_before(Iterator it);
};
\end{minted}
\end{listing}


\begin{listing}[!ht]
\caption{Definição do nó de uma lista duplamente encadeada}
\begin{minted}{C++}
template <typename T> 
class List<T>::Node{
    public:
    T val;
    Node * next;
    Node * prev;
    Node() : next(nullptr), prev(nullptr){}
    Node(T val) : next(nullptr), prev(nullptr){}
    Node(T val, Node * prev, Node * next) : val(val), prev(prev), next(next) {}    
};


\end{minted}
\end{listing}



\begin{listing}[!ht]
\caption{Definição do iterador da lista duplamente encadeada}
\begin{minted}{C++}
template <typename T> 
class List<T>::Iterator{
    private:
        Node *atual;
    public:
        Iterator() : atual(nullptr){}
        Iterator(Node * atual) : atual(atual){ }
        Iterator next(){ return Iterator( atual->next); }
        T & value() { return atual->val; }
        bool operator!=(const Iterator & other) const{
            return atual != other.atual;
        }
        void insert_before(T x)
        {
            auto ptr = new Node(x, atual->prev, atual);
            atual->prev->next = ptr;
            atual->prev = ptr;
        }
        void erase_before()
        {
            auto ptr = atual->prev;

            ptr->prev->next = atual;
            atual->prev = ptr->prev;

            delete ptr;

        }
};
\end{minted}
\end{listing}



\begin{listing}[!ht]
\caption{Definição de algumas operações da lista encadeada}
\begin{minted}{C++}
template <typename T> 
void List<T>::remove_front(){
    if(count > 0){
        auto ptr = head->next;
        head->next = ptr->next;
        ptr->next->prev = head;
        delete ptr;
        this->count--;
    }
}
template <typename T> 
void List<T>::remove_back(){
    
    if(count > 0){
        auto last = past_last->prev;
        past_last->prev = last->prev;
        last->prev->next = past_last;
        delete last;
        this->count--;
    }
}
template <typename T> 
void List<T>::insert_front(T x){
    auto ptr = new Node(x, head, head->next);
    head->next->prev = ptr;
    head->next = ptr;
    this->count++;
}
template <typename T> 
void List<T>::insert_back(T x){
    auto ptr = new Node(x, past_last->prev, past_last);
    past_last->prev->next = ptr;
    past_last->prev = ptr;
    this->count++;
}

\end{minted}
\end{listing}


