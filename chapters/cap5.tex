% The end to end process: second chapter.

\chapter{Listas sequencias com tamanho dinâmico}

Listas sequencias são tipos de dados abstratos representando vetores que podem ter seu tamanho alterado. Como os vetores, as listas sequenciais utilizam regiões de memórias contíguas para seus elementos, o que significa que seus elementos também podem ser acessados por meio de deslocamento em ponteiros para seus elementos com a mesma eficiência dos vetores. Mas ao contrário dos vetores, o seu tamanho pode mudar dinamicamente. Internamente, a lista sequencial usa um vetor alocado dinamicamente para armazenar seus elementos. Esse vetor pode precisar ser realocado para aumentar de tamanho quando novos elementos forem inseridos, o que implica em alocar um novo vetor e mover todos os elementos para ele.

Em geral, as listas sequenciais podem utilizar um espaço extra para acomodar um possível crescimento e assim a lista sequencial pode ter uma capacidade atual maior que a memória necessária para conter seus elementos.

Portanto, comparado com os vetores, as listas sequenciais consomem mais memória contudo elas possuem a habilidade de crescimento dinâmico.

\section{Programa Driver}

\begin{minted}{C++}
#include <stdio.h>
#include <cstdlib>
#include <iostream>

#include "ListaSequencial.hpp"

using namespace std;

int main(){

    ListaSequencial <int> v;

    v.push(5);

    cout << &v[0] << endl;
    cout << "size: " << v.size() << endl;
    cout << "capacity: " << v.getcapacity() << endl;
    
    v.push(7);
    cout << &v[0] << endl;
    cout << "size: " << v.size() << endl;
    cout << "capacity: " << v.getcapacity() << endl;
    
    v.push(9);
    cout << &v[0] << endl;
    cout << "size: " << v.size() << endl;
    cout << "capacity: " << v.getcapacity() << endl;
    
    v.push(10);
    cout << &v[0] << endl;
    cout << "size: " << v.size() << endl;
    cout << "capacity: " << v.getcapacity() << endl;
    
    v.push(11);
    cout << &v[0] << endl;
    cout << "size: " << v.size() << endl;
    cout << "capacity: " << v.getcapacity() << endl;
    
    
   
    cout << v << endl; // operator <<


    cout << "using operator []" << endl;
    int n = v.size();
    for(int i = 0; i < n; i++){
        cout << v[i] << endl; //operator []
    }

    cout << "using iterator, begin, end, prefix increment, *" << endl;
    ListaSequencial<int>::Iterator it;
    for(it = v.begin(); it != v.end(); ++it){
        cout << *it << endl;
    }

    cout << "using iterator, begin, end, posfix increment, *" << endl;
    for(it = v.begin(); it != v.end(); it++){
        cout << *it << endl;
    }




}

/*
0x55a5a02b8eb0
size: 1
capacity: 1
0x55a5a02b92e0
size: 2
capacity: 2
0x55a5a02b8eb0
size: 3
capacity: 4
0x55a5a02b8eb0
size: 4
capacity: 4
0x55a5a02b9300
size: 5
capacity: 8
[5 7 9 10 11]
using operator []5
7
9
10
11
using iterator, begin, end, prefix increment, *
5
7
9
10
11
using iterator, begin, end, posfix increment, *
5
7
9
10
11



*/
\end{minted}


\section{Cabeçalho}


\begin{minted}{C++}
#ifndef LISTASEQUENCIAL_HPP
#define LISTASEQUENCIAL_HPP

#include <stdexcept>

template <typename T> class ListaSequencial;

using namespace std;
template <typename T> class ListaSequencial
{
 
    // arr is the integer pointer
    // which stores the address of our vector
    T* arr;
 
    // capacity is the total storage
    // capacity of the vector
    int capacity;
 
    // current is the number of elements
    // currently present in the vector
    int current;
 
public:
    
    
    class Iterator{

        private:
            const ListaSequencial <T> * pLista;
            int m_index = -1;

        public:
            Iterator(){
                pLista = nullptr;
                int m_index = -1;
            }      

            Iterator(const ListaSequencial<T> * lista, int nIndex){
                pLista = lista;
                m_index = nIndex;
            }

            Iterator & operator++(){
                ++m_index;
                return *this;
            }

            Iterator operator++(int){ //posfix version
                Iterator it(pLista, m_index);
                ++(*this);
                return it;
            }

            const T & operator*() const{
                return pLista->operator[](m_index);

            } 

            bool operator!=(const Iterator & other) const{
                return m_index != other.m_index;
            }


    };

    
    // Default constructor to initialise
    // an initial capacity of 1 element and
    // allocating storage using dynamic allocation
    ListaSequencial()
    {
        arr = new T[1];
        capacity = 1;
        current = 0;
    }

    

 
    // Function to add an element at the last
    void push(T data)
    {
 
        // if the number of elements is equal to the
        // capacity, that means we don't have space to
        // accommodate more elements. We need to double the
        // capacity
        if (current == capacity) {
            T* temp = new T[2 * capacity];
 
            // copying old array elements to new array
            for (int i = 0; i < capacity; i++) {
                temp[i] = arr[i];
            }
 
            // deleting previous array
            delete[] arr;
            capacity *= 2;
            arr = temp;
        }
 
        // Inserting data
        arr[current] = data;
        current++;
    }
 
    // function to add element at any index
    void push(T data, int index)
    {
 
        // if index is equal to capacity then this
        // function is same as push defined above
        if (index == capacity)
            push(data);
        else
            arr[index] = data;
    }
 
    const T & operator[](int index) const{
        if (index >= capacity){
            throw std::runtime_error("index out of range");
        }
        return arr[index];
    }
 
    // function to delete last element
    void pop() { current--; }
 
    // function to get size of the vector
    int size() const { return current; }
 
    // function to get capacity of the vector
    int getcapacity() { return capacity; }
 
    Iterator begin() const
    {
        return ListaSequencial<T>::Iterator(this, 0);
    }

    Iterator end() const 
    {
        return ListaSequencial<T>::Iterator(this, current);
    }



};

template <typename T>
ostream& operator<<(ostream &output, const ListaSequencial<T>& v)
{
    int n = v.size();
    output << "[";
    if( n > 0){
        output << v[0];
        for(int i = 1; i < n; i++)
            output << " " << v[i];
    }
    output << "]";
    return output;
}

#endif
\end{minted}


\section{Exercício}


Dado uma ListaSequencial de inteiros nums e um inteiro val remova todas as ocorrências de val da ListaSequencial.

Você deve colocar o resultado na primeira parte do array nums. Mais formalmente, se houver k elementos após a remoção dos elementos iguais a val, os primeiros k elementos de nums devem conter o resultado final. Não importa o que você deixe além dos primeiros k elementos.

referência: \url{https://leetcode.com/problems/remove-element/}

\begin{minted}{C++}
#include <stdio.h>
#include <cstdlib>
#include <iostream>

#include "ListaSequencial.hpp"

using namespace std;

int removeElement(ListaSequencial<int> v, int val){

    int n = v.size();
    int k = 0;
    for(int i = 0; i < n; i++){
        if(v[i] != val){
            v[k++] = v[i];
        }
    }

    return k;

}

int main(){

    ListaSequencial <int> v;
    v.push(4);
    v.push(5);
 	v.push(6);
    v.push(5);
    v.push(3);

    int k = removeElement(v, 5);

    for(int i = 0; i < k ; i++){
        cout << v[i] << endl;
    }


}

\end{minted}


\url{https://www.hackerearth.com/practice/data-structures/arrays/1-d/practice-problems/algorithm/minimum-additions-0142ac80/}



