\chapter{Fila}

Na ciência da computação, uma fila é um estrutura de dados que mantém uma coleção de elementos de maneira que a operação de dequeue remove o elemento mais antigo da fila e operação enqueue adiciona um novo elemento na fila preservando a ordem de tempo na fila. Uma maneira fácil de implementar é manter descritores para o ínicio e o fim da fila. Esses descritores serão importantes para manter a corretude das operações enqueue e dequeue.

Na linguagem C++, a fila pode ser implementada estendendo a classe list (que por sua vez são implementadas como listas duplamente encadeadas) implementando as operações push\_back, pop\_front, front.

\begin{minted}{C++}
template <typename T> 
class queue : private list <T>
{
  public:
    using base_type = list <T> ; 
  void  enqueue ( const T& x ) { base_type::push_back( x ); }
  const T& front  ()             { return base_type::front(); }
  void  dequeue  ()             {  base_type::pop_front(); }
  bool  empty()             { return base_type::empty(); }
};
\end{minted}

Uma desvantagem de implementar a fila herdando da STL list é que a STL list é implementada como uma lista duplamente encadeada. Podemos implementar a fila usando uma lista encadeada simples guardando um ponteiro para o primeiro e o último elemento da lista. Na seção \ref{sec::queue_single_linked_list}, apresentamos a implementação da fila usando uma lista encadeada simples.


De igual modo, a fila também pode ser implementada estendendo a classe vector da linguagem C++. 

\begin{minted}{C++}

template <typename T> 
class queue : private vector <T>
{
  public:
    using base_type = vector <T> ; 
  void  enqueue ( const T& x ) { base_type::push_back( x ); }
  const T& front  ()             { return base_type::front(); }
  void  dequeue  ()             {  base_type::erase( base_type::begin() ); }
  bool  empty()             { return base_type::empty(); }
};

\end{minted}

Contudo, nessa implementação, a operação dequeue passa a ter complexidade linear. Lembrando que no vector a operação remove o primeiro elemento do vetor possui complexidade linear. Na seção \ref{sec::circular_ queue}, apresentamos uma implementação de uma fila circular utilizando um vetor de tamanho fixo em que a operação dequeue é realizada em tempo constante.









\section{Fila com lista encadeada simples com dois descritores}
\label{sec::queue_single_linked_list}

Nessa implementação, iremos implementar a fila utilizando dois ponteiros para o primeiro elemento e para o último. Na classe Node, teremos apenas um ponteiro para o próximo. Lembrando que na fila não temos uma travessia bidirecional.

\begin{minted}{C++}
template <typename T> 
class queue{
    private:
        class Node;
        Node * first;
        Node * last;
        int cnt;
    public:
        queue();
        ~queue();
        bool empty();
        void enqueue(T key);
        void dequeue();
        T front();
        int size();
};

template <typename T> 
class queue<T>::Node{
    public:
    T key;
    Node * next; 
    Node(T key, Node * next = nullptr) : key(key), next(next) {};
};
\end{minted}


As operações da lista podem ser implementadas facilmente:

\begin{minted}{C++}
template <typename T> 
queue<T>::queue()
{
    first = last = nullptr;
    cnt  = 0;
}
template <typename T> 
queue<T>::~queue(){
    auto ptr = first;
    while(ptr){
        auto temp = ptr;
        ptr = ptr->next;
        delete temp;
    }
    first = last = nullptr;
    cnt = 0;
}

template <typename T> 
bool queue<T>::empty()
{
    return cnt == 0;
}

template <typename T> 
int queue<T>::size()
{
    return cnt;
}

template <typename T> 
void queue<T>::enqueue(T key){
    auto ptr = new Node(key);
    if(last != nullptr)
        last->next = ptr;
    else
        first = ptr;
    last = ptr;
    cnt++;
}

template <typename T> 
void queue<T>::dequeue()
{
    if( first != nullptr)
    {
        auto temp = first;
        first = first->next;

        if(first == nullptr)
            last = nullptr;
        
        delete temp;

        cnt--;
    }
}

template <typename T> 
T queue<T>::front()
{
    if (first)
        return first->key;
    else
        return T();
}

\end{minted}


\section{Fila Circular}
\label{sec::circular_ queue}

\begin{minted}{C++}
template <typename T> 
class circular_queue{
    private:
        T * data;
        int length;
        int capacity;
        int first;     
    public:
        circular_queue(int total);
        ~circular_queue();
        bool empty();
        void enqueue(T key);
        T dequeue();
        T front();
        bool full();
        int size();
};
\end{minted}

Construtor da classe
\begin{minted}{C++}
template <typename T> 
circular_queue<T>::circular_queue(int total)
{
    data = new T[total];
    capacity = total;
    first = 0;
    length = 0;
}
\end{minted}

Método empty e full
\begin{minted}{C++}
template <typename T> 
bool circular_queue<T>::empty()
{
    return length == 0;
}

template <typename T> 
bool circular_queue<T>::full()
{
    return length == capacity;
}

\end{minted}

Método enqueue
\begin{minted}{C++}
template <typename T> 
void circular_queue<T>::enqueue(T key)
{
    if( length == capacity )
    {
        cerr << "fila cheia";
        return ;
    }
    else
    {
        int fim = (first + length ) % capacity;
        data[fim] = key;
        length++;
    }
}
\end{minted}


Método dequeue
\begin{minted}{C++}
template <typename T> 
T circular_queue<T>::dequeue()
{
    if( empty() )
    {
        cerr << "fila vazia";
        return T();
    }
    else
    {
        T val = data[first];
        first = (first + 1)%capacity;
        length--;
        return val;
    }
}

\end{minted}

\section{Revisitando o Josephus Problem}


\begin{minted}{C++}
int josephus_queue(int n, int k){

  queue <int> q;

  for(int i = 0; i < n; i++) q.push(i);

  while( q.size() > 1){
    //skip k-1
    for(int i = 0; i < k-1; i++){
      auto x = q.front();
      q.push(x);
      q.pop();
    }

    //cout << "kill " << q.front() << endl;
    q.pop();
  }  

  return q.front();

}
\end{minted}


\begin{minted}{C++}
int josephus_circular_queue(int n, int k)
{
  circular_queue <int> q(n);

  for(int i = 0; i < n; i++) q.enqueue(i);

  while( q.size() > 1){
    //skip k-1
    for(int i = 0; i < k-1; i++){
      auto x = q.front();
      q.dequeue();
      q.enqueue(x);
    }

    //cout << "kill " << q.front() << endl;
    q.dequeue();
  }  

  return q.front();

}

\end{minted}

\begin{center}
\begin{tabular}{lll}
\hline
$(n,k)$ & circular\_queue & queue \\
\hline
$(10^6, 6)$ & 0.104795 & 0.165802\\
$(10^7, 6)$ & 0.715600 & 1.658800\\
$(10^7, 6)$ & 0.715600 & 1.658800\\
$(10^8, 6)$ & 6.759660 & 16.751390\\
  
\end{tabular}
\end{center}



